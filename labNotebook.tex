\documentclass[10pt,a4paper]{article}
\usepackage[utf8]{inputenc}
\usepackage[french]{babel}
\usepackage[T1]{fontenc}
\usepackage{amsmath}
\usepackage{amsfonts}
\usepackage{amssymb}
\usepackage{graphicx}
\usepackage{hyperref}
\usepackage[left=2cm,right=2cm,top=2cm,bottom=2cm]{geometry}
\usepackage{siunitx}
\author{Baptiste Rouger}
\title{Lab Notebook MSC}

\begin{document}
\maketitle

\tableofcontents

\newpage

\section{15 Jan 2018}

\begin{itemize}
        \item Début d'installation sur le PC
        \item Cassage de ArchLinux
        \item Rangement de la salle de manip et mise en place de la salle de manip
                \begin{itemize}
                        \item Raccourcissement des plus longues barres de la cage qui gênaient.
                        \item Réorganisation de la salle
                \end{itemize}
        \item Lecture de la review de Mathieu
\end{itemize}
\noindent Comment la température ambiante influence la fermeture des feuilles? \\
Comment la fermeture des feuilles affecte la température de celles-ci?

\section{16 Jan 2018}
\begin{itemize}
        \item Installation de Debian sur le PC
        \item Installation des logiciels importants sur le PC
        \item Mise en place de la première manip test pour la \textbf{nutation}: début à \textbf{15h52}, fin à \textbf{10h42} le 17 Jan 2018. Les données sont situées sur Alfred: /mnt/data/manip/Baptiste/test\_16-01-2018.J'ai utilisé la plante "Abby" pour réaliser cette manip. Photo toutes les 90 secondes.
\end{itemize}

\section{17 Jan 2018}
\begin{itemize}
        \item Arrêt de la manip \textbf{test\_16-01-2018} à \textbf{10h42}
        \item Arrosage des plantes
        \item Récupération des données de la manip
        \item Création du film à partir des données de la manip (\textsc{Section}~\ref{film})
        \item Réalisation du script \textbf{analysisScript.py} qui, à partir de photos stockées dans un dossier, réalise la timeline d'une ligne de pixel et la converti en image binaire
\end{itemize}

Le lien pour la vidéo: \url{http://uptobox.com/5x80eimcd7xu}\\

Le résultat du script \textbf{analysisScript.py} :\\
\includegraphics[width = \linewidth]{imTLBWv1.jpg}

\section{18 Jan 2018}
\begin{itemize}
        \item Début d'une deuxième manip pour observer la nutation (sur un plus long terme). Début à environ 12h30 \textbf{(heure pc décalée????)}. Il fait \SI{24.2}{\celsius} dans la pièce, 49\% d'humidité. On utilise la plante "Béa". Les numéro de protocole sont:
                \begin{itemize}
                        \item arduino: 21608
                        \item photos: 21821
                \end{itemize}
        \item Création du script appelé \textbf{anSc3.py} qui utilise la librairie skimage pour analyser les images et réaliser le graphe de mouvement.

\end{itemize}

\section{19 Jan 2018}
Je me suis penché sur le problème du maintien de la température et de l'humidité de la salle de manip avec Arduino (ainsi que la photopériode). J'ai pris en compte la mesure de ces données. Schéma :\\
\includegraphics[width=0.6\linewidth]{schema_Arduino.png}

\section{20 Jan 2018}
J'ai refait le focus sur la jeune feuille, il n'était évidemment plus bon au bout de ces 2 jours et demi. $\Rightarrow$ Il serait intéressant de voir si l'auto focus pourrait compenser intelligemment cela.\\

Pour les scripts, je vais essayer de trouver la position du centre de la tige en utilisant une moyenne pondérée par le nombre de pixels consécutifs\\

J'ai modifié les scripts : je les ai séparé en deux : \texttt{TL.py} sert à créer l'image de la timeline, \texttt{Trajec.py} sert à récupérer la position moyenne de la tige. On peut entre les deux scripts retailler l'image.





\newpage
\begin{center}
        {\Huge Protocoles}
\end{center}
\appendix

\section{Creation du film à partir des images\label{film}}
\begin{itemize}
        \item On utilise Thunar pour renommer nos fichier pour que leurs noms soient une suite numérotée ininterrompue (eg. 001.jpg, 002.jpg, etc)
        \item On utilise la commande \texttt{ffmpeg -framerate 40 -i \%03d.jpg -c:v libx264 -profile:v high -crf 20 -pix\_fmt yuv420p output.mp4}
\end{itemize}
Les fichiers à utiliser sont données par l'option \texttt{-i}, on peut changer le framerate (ici 40 images par secondes).
\end{document}
