\documentclass[10pt,a4paper]{article}
\usepackage[utf8]{inputenc}
\usepackage[french]{babel}
\usepackage[T1]{fontenc}
\usepackage{amsmath}
\usepackage{amsfonts}
\usepackage{amssymb}
\usepackage{graphicx}
\usepackage{hyperref}
\usepackage[left=2cm,right=2cm,top=2cm,bottom=2cm]{geometry}
\author{Baptiste Rouger}
\title{Lab Notebook MSC}

\begin{document}
\maketitle

\tableofcontents

\newpage

\section{15 Jan 2018}

\begin{itemize}
\item Début d'installation sur le PC
\item Cassage de ArchLinux
\item Rangement de la salle de manip et mise en place de la salle de manip
	\begin{itemize}
	\item Raccourcissement des plus longues barres de la cage qui gênaient.
	\item Réorganisation de la salle
	\end{itemize}
\item Lecture de la review de Mathieu
\end{itemize}
~\\
\noindent Comment la température ambiante influence la fermeture des feuilles? \\
Comment la fermeture des feuilles affecte la température de celles-ci ?

\section{16 Jan 2018}
\begin{itemize}
\item Installation de Debian sur le PC
\item Installation des logiciels importants sur le PC
\item Mise en place de la première manip test pour la \textbf{nutation} : début à \textbf{15h52}, fin à \textbf{10h42} le 17 Jan 2018. Les données sont situées sur Alfred : /mnt/data/manip/Baptiste/test\_16-01-2018
\end{itemize}

\section{17 Jan 2018}
\begin{itemize}
\item Arrêt de la manip \textbf{test\_16-01-2018} à \textbf{10h42}
\item Arrosage des plantes
\item Récupération des données de la manip
\item Création du film à partir des données de la manip (\textsc{Section}~\ref{film})
\item Réalisation du script \textbf{analysisScript.py} qui, à partir de photos stockées dans un dossier, réalise la timeline d'une ligne de pixel et la converti en image binaire
\end{itemize}

Le lien pour la vidéo : \url{http://uptobox.com/5x80eimcd7xu}\\

Le résultat du script \textbf{analysisScript.py} :\\
\includegraphics[width = \linewidth]{imTLBWv1.jpg}


\newpage
\begin{center}
{\Huge Protocoles}
\end{center}
\appendix

\section{Creation du film à partir des images \label{film}}
\begin{itemize}
\item On utilise Thunar pour renommer nos fichier pour que leurs noms soient une suite numérotée ininterrompue (eg. 001.jpg, 002.jpg...)
\item On utilise la commande \texttt{ffmpeg -framerate 40 -i \%03d.jpg -c:v libx264 -profile:v high -crf 20 -pix\_fmt yuv420p output.mp4}
\end{itemize}
Les fichiers à utiliser sont données par l'option \texttt{-i}, on peut changer le framerate (ici 40 images par secondes).
\end{document}